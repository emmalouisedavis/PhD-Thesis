\documentclass[a4paper,12pt]{article}
\usepackage[margin=1in]{geometry}
\usepackage{enumitem}
\usepackage{amsmath}
\usepackage{mathtools}
\usepackage{amssymb}
\usepackage{graphicx}
\usepackage{placeins}
\usepackage{listings}
\usepackage{tkz-graph}
\usepackage{lipsum}
\GraphInit[vstyle = Shade]
\tikzset{
  LabelStyle/.style = { rectangle, rounded corners, draw,
                        minimum width = 2em,
                        text = red, font = \bfseries },
  VertexStyle/.append style = { inner sep=5pt,
                                font = \Large\bfseries},
  EdgeStyle/.append style = {->, bend left} }
\usepackage{xcolor}
\lstset{language=Matlab,%
    %basicstyle=\color{red},
    breaklines=true,%
    morekeywords={matlab2tikz},
    keywordstyle=\color{blue},%
    morekeywords=[2]{1}, keywordstyle=[2]{\color{black}},
    identifierstyle=\color{black},%
    stringstyle=\color{mylilas},
    commentstyle=\color{gray},%
    showstringspaces=false,%without this there will be a symbol in the places where there is a space
    numbers=left,%
    numberstyle={\tiny \color{black}},% size of the numbers
    numbersep=9pt, % this defines how far the numbers are from the text
    emph=[1]{for,end,break},emphstyle=[1]\color{red}, %some words to emphasise
    %emph=[2]{word1,word2}, emphstyle=[2]{style},    
}
\usepackage{wrapfig}

\begin{document}

\title{Vector model notation}
\author{Emma Davis}
\date{\today}
\maketitle

\subsection*{Useful statistics}

Entomological innoculation rate (EIR):

\begin{equation}
E = maz
\end{equation}

\noindent Vectorial capacity:

\begin{equation}
V = \frac{ma^2p^v}{-\ln(p)} = \frac{ma^2}{g}e^{-gv}
\end{equation}

\noindent Basic reproductive number:

\begin{equation}
R_0 = \frac{ma^2bc}{gr}e^{-gv} = \frac{ma^2bc}{-\ln(p)r}p^v = \frac{Vbc}{r}
\end{equation}

\subsection*{Model results}

We can calculate the total number of diseased vectors (exposed and infectious), $D$:

\begin{eqnarray}
D := Y+Z = \frac{\kappa KE_0}{(1-K)(1-K+\kappa K)}\\
E_0 = \frac{\beta(1-\theta)}{\pi_2(q_1+q_2)+g}
\end{eqnarray}

\noindent From this we can get the total number of infectious vectors:

\begin{equation}
Z = K^nD\,,
\end{equation}

and hence directly calculate our useful statistics. For example, EIR is given by:

\begin{equation}
E = maK^n\bigg(\frac{\kappa KE_0}{(1-K)(1-K+\kappa K)}\bigg)\frac{1}{M}
\end{equation}

\noindent Still need to comput how the following parameters will be affected by vector control:

\begin{itemize}
\item $a=Q/\delta$ -- $\delta$ will be affected by bednets. Need to calculate in terms of $q_1$ etc.
\item $n$ will change with $a$ -- lower bite rate will give longer cycles and hence incubation period will be fewer cycles. Always round up!
\item $m$ -- fix \# of hosts and recalcuate ratio for vector population (i.e. just scale with reduction in vector population for all control measures)
\item $g$ -- death rate will be affected by IRS and bednets
\end{itemize}

In the absence of interventions the mean feeding cycle length is,

\begin{equation}
\delta = 1/\pi_1+1/\pi_2+1/\pi_3+1/\pi_4\,,
\end{equation}

where $1/\pi_2$ is the average time to hunt and take a blood meal. IRS and larvicides won't impact hunting time, but the repelling effect of bednets will result in some vectors taking longer to move from emerged to fed.

If we assume a repelled vector begins the hunting process from scratch, then the expected time taken to successfully feed will be equal to the time taken to feed given a successful first attempt plus the expected time taken to feed scaled by the proportion of vector that repeat on any given attempt.

\begin{eqnarray}
\mathbb{E}[\mbox{Time to feed}] &=& \mathbb{E}[\mbox{Time }|\mbox{ Successful attempt}] + \mathbb{P}[\mbox{Repeat}]\mathbb{E}[\mbox{Time to feed}]\\
\mathbb{E}[\mbox{Time to feed}] &=& \frac{1}{\pi_2} + Q\omega(1-\sigma-\nu)\mathbb{E}[\mbox{Time to feed}]\\
\mathbb{E}[\mbox{Time to feed}] &=& \frac{1}{\pi_2(1-Q\omega(1-\sigma-\nu))}\\
\end{eqnarray}

Now we can express overall feeding cycle length, $\delta$, in terms of bednet parameters:

\begin{eqnarray}
\delta &=& \frac{1}{\pi_2(1-Q\omega(1-\sigma-\nu))} + \frac{1}{\pi_3}+\frac{1}{\pi_4}+\frac{1}{\pi_1}\\
&=& \frac{(\pi_1\pi_3+\pi_1\pi_4+\pi_3\pi_4)\pi_2(1-Q\omega(1-\sigma-\nu))+\pi_1\pi_3\pi_4}{\pi_1\pi_2\pi_3\pi_4(1-Q\omega(1-\sigma-\nu))}
\end{eqnarray}

and the human blood feeding rate is given by:

\begin{eqnarray}
a &=& Q\bigg(\frac{1}{\pi_2(1-Q\omega(1-\sigma-\nu))} + \frac{1}{\pi_3}+\frac{1}{\pi_4}+\frac{1}{\pi_1}\bigg)^{-1}\\
&=& Q\frac{\pi_1\pi_2\pi_3\pi_4(1-Q\omega(1-\sigma-\nu))}{(\pi_1\pi_3+\pi_1\pi_4+\pi_3\pi_4)\pi_2(1-Q\omega(1-\sigma-\nu))+\pi_1\pi_3\pi_4}
\end{eqnarray}

The ratio of vectors to humans $m$, can be scaled by changes in the mosquito population ($m=M/H$), where

\begin{equation}
M = \sum_{i=0}^{\infty}E_0K^i = \frac{E_0}{1-K}
\end{equation}

and $K$ describes the probability of surviving each feeding cycle, with dependence on vector control parameters included in $E_0$ and $K$.

The death rate will depend on IRS and bednet usage. We can relate the probability of a vector surviving one feeding cycle, $K$, to a per cycle death rate $-ln(K)$, then we have

\begin{equation}
g = \frac{-ln(K)}{\delta}
\end{equation}

as the instanenous daily death rate.

\FloatBarrier

\begin{table*}[h]
\caption{Standard parameter names for vector models (Smith et al 2012).}% title of Table
\vspace{.1cm}
\centering % used for centering table
\begin{tabular}{l c c}% centered columns (4 columns)
\hline\hline                        %inserts double horizontal lines
Definition & Parameter & Notes \\ [0.5ex]% inserts table 
%heading
\hline                  % inserts single horizontal line
Population density of humans & $H$ &  \\
Population density of mosquitoes & $M$ &  \\
Number of infected humans & $X$ &  \\
Number of exposed vectors & $Y$ & $N_E$ \\
Number of infectious vectors & $Z$ & $N_I$ \\
Human blood feeding rate (prop vectors that feed on humans/day) & $a$ & \\
Probability vector survives one day & $p$ & \\
Instantaneous vector death rate & $g$ & $g=-\ln(p)$  \\
Intrinsic incubation period (in humans) & $u$ & \\
Extrinsic incubation period (in vectors) & $v$ & \\
Human recovery rate & $r$ & \\
Prob vector infected after biting infected human & $c$ & \\
Prob infectious bite infects a human & $b$ & \\
Blood feeding rate (all prey) & $f$ & \\
Fraction of blood meals on humans & $Q$ & $a=fQ$ \\
Mean feeding cycle length & $\delta$ & $a=Q/\delta$\\
Prevalence of malaria in humans & $x$ & \\
Fraction of exposed vectors & $y$ & \\
Fraction of infectious vectors & $z$ & \\
Ratio of vectors to humans & $m$ & $=M/H$ \\
Human biting rate (\# bites per human per day) & $HBR$ & $=ma$\\
EIR (\# infectious bites per human per day) & $E$ & $=maz$ \\
The average mosquito lifespan & $1/g$ & \\
Stability index (\# human bites per vector across its life) & $S$ & $=a/g$\\
Probability vector survives to infectiousness & $P$ & $=e^{-gv}$\\
Probability vector becomes infected after human blood feed & $\kappa$ & $=cx$\\
Vectorial capacity & $V$ & \\
Basic reproductive number & $R_0$ & \\
Effective reproductive number under control & $R_c$ & \\
Critical vector density required to sustain transmission & $m'$ & \\
      % [1ex] adds vertical space
\hline%inserts single line
\end{tabular}
\label{table:param}% is used to refer this table in the text
\end{table*}

\begin{table*}[t]
\caption{Additional parameter names required for this model.}% title of Table
\vspace{.1cm}
\centering % used for centering table
\begin{tabular}{l c c}% centered columns (4 columns)
\hline\hline                        %inserts double horizontal lines
Definition & Parameter & Notes \\ [0.5ex]% inserts table 
%heading
\hline                  % inserts single horizontal line
Rate of moving from Exposed to Fed & $\pi_2$ & \\
Rate of moving from Fed to Resting & $\pi_3$ & \\
Rate of moving from Resting to Ovipositing & $\pi_4$ & \\
Rate of moving from Ovipositing to Exposed & $\pi_1$ & \\
Emergence rate (from larval stages) & $\beta$ & \\
Linear reduction in emergence rate due to larvicides & $\theta$ & \\
Additional IRS-induced death rate & $\gamma$ & \\
Bednet coverage & $\omega$ & \\
Probability successful feed in presence of bednet & $\sigma$ & \\
Probability death caused by bednet & $\nu$ & \\
Probability successful feed on single attempt & $q_1$ & $=1-Q\omega(1-\sigma)$ \\
Proabability death on a single feeding attempt & $q_2$ & $=Q\omega\nu$\\
Probability a vector survives one feeding cycle & $K$ & previously $C$\\
The number of newly emerged null-parous vectors & $B_0$ &\\
Extrinsic incubation period (in number of feeding cycles) & $N$ & \\
Probability of vector infection from one successful bite & $\hat{p}$ & $=xc$ \\
\hline%inserts single line
\end{tabular}
\label{table:param2}% is used to refer this table in the text
\end{table*}

\end{document}

%%
%% End of file `ejemplo latex RIAI.tex'.