The term `Neglected Tropical Diseases' (NTDs) first came into common use during World Health Organization (WHO) meetings in Berlin in 2003 and 2004 \cite{WHO_NTD}. The subsequent creation of a WHO Department for Control of Neglected Tropical Diseases led to a series of events, concluding in the publication of a WHO Road Map in 2012 describing global targets for the period 2012-2020 \cite{Roadmap} and a meeting in London in the same year entitled `Uniting to combat NTDs: ending the neglect and reaching the 2020 goals'. This meeting was attended by Bill Gates, the WHO Director-General, the CEOs of major pharmaceutical companies, senior government officials from endemic and donor countries, and representatives of academic institutions and civil society. The outcome was the London Declaration: an agreement from twelve of the world’s biggest pharmaceutical companies to ensure sustained drug donations to help meet the control and elimination goals set by WHO \cite{Allen2012}. 

There were originally seventeen diseases under the NTD umbrella, representing a diverse range of infections which are common in low-income populations in developing countries across the Americas, Africa and Asia \cite{Roadmap}. These populations also often had very poor, or no, access to local health care. In 2017 three additional diseases were added, bringing the total to twenty \cite{WHO_NTD}. The additional funding and global support that has been mobilised due to the collective rebranding of these diseases has led to huge progress in a number of instances \cite{rebollo2015,Njenga2011,Minetti2019}. Progress in some diseases, such as lymphatic filariasis and trachoma, has been so substantial that the focus has shifted from control to elimination, with the new 2030 goals reflecting this \cite{WHO2017_GPELF,NTDMC2019}.

Mathematical models have long played a role in public health planning for major diseases such as malaria, with Ross' initial model of mosquito-borne disease dating back to 1908 \cite{Ross1908}, and Daniel Bernoulli describing the first mathematical model of small pox in 1766 \cite{Dietz2000bernoulli}. However, not all NTDs share this long history of modelling involvement. For example, the first specific models describing lymphatic filariasis transmission were developed in the 1990s \cite{Rochet1990,Plaisier1998}. There have be substantial leaps in theory in recent years and the formation of groups such as the NTD Modelling Consortium and the DeWorm3 project, both funded by the Bill and Melinda Gates Foundation, has resulted in increased engagement between mathematicians, the WHO and public health professionals \cite{NTDMC2019}. 

This context means it is increasingly important to ensure models are biologically sensible and well parameterised \cite{Davis2019}. In particular, when modelling interventions such as mass drug administration (MDA -- regular population-wide use of chemotherapy) there are a number of global and local factors that must be considered. In this thesis I challenge some of the assumptions made by current model frameworks and investigate the impact of a range of factors on modelling outcomes.

I have not included a separate literature review chapter, due to individual literature review content being included in each chapter as follows:

\begin{itemize}
    \item In \textbf{Chapter 1}, I cover the global context of NTDs and a review of mathematically modelling methods for diseases, macro-parasites and mosquito dynamics.
    \item In \textbf{Chapter 2}, I describe the global burden of soil-transmitted helminths (STH) and review the evidence in the literature for seasonal influencers on transmission. 
    \item In \textbf{Chapter 3}, I introduce the biology of lymphatic filariasis (LF), discuss currently used models of transmission and review the empirical evidence behind key parasite life-cycle parameters. 
    \item In \textbf{Chapter 4}, I review the modelling literature on mosquito-borne diseases, focusing on LF and drawing comparisons with malaria. I also discuss present evidence for the use of vector control as an intervention tool for both diseases.
    \item In \textbf{Chapter 5}, I review the literature surrounding xenomonitoring as a tool for LF surveillance.
\end{itemize}

\section{Background}

\subsection{Global context of neglected tropical diseases (NTDs)}

NTDs affect more than one billion people across 149 countries \cite{WHO_NTD} and cost billions of dollars every year. A subset of these are caused by helminth, or macro-parasitic worm, infections, including STH, LF, onchocerciasis, Guinea worm and schistosomiasis. There are two main subdivisions of helminths: nematodes, including intestinal and filarial worms, and flatworms, including schistosomes \cite{Hotez2008}. Helminths are particularly persistent due to their ability to modulate their host's immune response \cite{Maizels2018} and in 2006 were estimated to affect around a quarter of the world's population \cite{Bethony2006}.

Helminth infections are characterised by the intensity of disease, rather than simply presence or absence. A higher parasite load will generally result in more severe symptoms and a higher risk of serious complications. For example, a high intensity \textit{Ascaris} infection, one of the STH parasites, could cause intestinal blockage or impair growth of children \cite{WHO}. Some infections are also associated with potential long-term complications, such as individuals with LF developing elephantiasis, which is associated with painful and irreversible swelling to the limbs, breasts or genitals \cite{WHO2019_FactSheet}. 

\subsection{Soil-transmitted helminths (STH)}

STH, or intestinal worms, are transmitted through excretion of eggs in the faeces of infected individuals. In areas that lack adequate sanitation and hygiene, particularly where there is poor access to secure latrine facilities, onward transmission is then caused by the resulting contamination of the soil \cite{Bethony2006}. Eggs can then be ingested through a number of routes, including in soil attached to unwashed vegetables, contaminated water sources and through children playing in or eating the soil \cite{WHO}. In addition, hookworm eggs hatch in the external environment and the resulting worm can pierce the skin, meaning infection can occur by walking barefoot through contaminated soil \cite{Bethony2006,Truscott2017}.

STH infections lead to nutritional impairment by reducing levels of iron and protein and increasing malabsorption of nutrients \cite{WHO}. School-aged children (SAC) and pre-school-aged children (pre-SAC) have generally been found to harbour the greatest burden of intestinal worms \cite{Hotez2008}, particularly for \textit{Ascaris} and \textit{Trichuris} \cite{Anderson2015}. Due to the labour intensive and unreliable nature of most diagnostic tools, the WHO recommended approach to controlling STH involves regular mass distribution of drugs in endemic and at-risk regions, otherwise known as MDA, with a focus on treating SAC  \cite{WHO2017STH}. However, recent evidence has suggested that community-wide MDA would be more appropriate in some settings \cite{Anderson2015,Farrell2018}.

\subsection{Lymphatic filariasis (LF)}

LF is transmitted through a complex life-cycle with an intermediary host. Developmental stages called microfilaria (mf) circulate in the blood of infective individuals and can be ingested by mosquitoes in the process of taking a blood meal. These mf then develop into a third-stage larvae (L3) in the mosquito, before migrating to the mosquito head and proboscis (\textit{viz.} mouth). When the mosquito takes future blood meals the larvae can then enter the skin of the host, eventually migrating into the lymphatic system and developing to maturity \cite{WHOLF}. As with many intestinal worm infections, sexual reproduction is required to produce mf, which will then circulate in the infected host's blood \cite{WHO2019_FactSheet,Anderson1992}. The most common parasite is \textbf{Wucheria bancrofti}, accounting for 90\% of all cases worldwide and the most common vectors are \textit{Anopheles spp.} and \textit{Culex spp.}, which are also known to commonly transmit malaria \cite{WHO2019_FactSheet}.

An individual infected with LF may not necessarily be mf positive, particularly if they have a single worm infection as this makes sexual reproductive impossible, and this can impact the utility of diagnostic methods. The most most commonly used test for diagnosing infection is a night blood smear test, where blood is drawn at night and inspected under a microscope for presence of mf, which can only detect infectious individuals. If blood is drawn during the day then the mf, which only circulate at night, are not usually detectable \cite{WHO2019_FactSheet}. However, antigen tests such as the immunochromatographic test (ICT) and the filariasis test strip (FTS) are increasingly being used in surveillance programs \cite{weil1997,Chesnais2016}. 

The most common intervention used to combat LF is MDA, as with STH, but vector control methods such as bednets and indoor residual spraying also have the potential to reduce transmission \cite{WHO2017_GPELF,rebollo2015}. In 1993 LF was earmarked as eradicable with current tools in a report by the International Task Force for Disease Eradication and in the years since the global focus has moved towards elimination rather than control \cite{WHO2017_GPELF}. Annual MDA is recommended by the WHO until a standardised survey indicates sufficiently low transmission to halt interventions. Following MDA cessation, three rounds of transmission assessment survey (TAS) have to be passed, indicating a microfilaria prevalence of $<$1\% or an antigen prevalence of $<$2\% \cite{WHO2017_Validation}. When a country has achieved these steps, they can prepare and submit a dossier to the WHO with the aim of being validated as having achieved elimination as a public health problem (EPHP).

\subsection{Mathematical modelling of disease}

In the early 20th century Ronald Ross was one of the first academics to characterise disease transmission using mathematical models \cite{Ross1911}, ultimately playing a key role in the development of the commonly-used compartmental SIR model. Variations on the SIR model have been used across mathematics and epidemiology to describe the dynamics of a huge range of different diseases \cite{Weiss2013}. This type of model is described by compartmentalising the population of interest into different categories representing their infection status; the SIR model considers individuals to belong to either a susceptible (S), infectious (I) or recovered/removed (R) class.

The dynamics of compartmentalised SIR-type models are described using rates of transmission, birth, death, recovery and other such processes and usually consist of a series of coupled differential equations. Other examples include the SIS model (susceptible, infectious, susceptible), which is often used to model diseases with minimal immunity consideration such as the common cold \cite{Banasiak2013}. There is also the SEIR model (susceptible, exposed, infectious, recovered), used when there is a delay between contracting the disease and infectivity, otherwise known as an incubation period \cite{Bolker1993}. Additional complex can be added in a number of ways, including adding further compartments, incorporating age-structure, or considering spatial dynamics.

A key epidemiological measure, which can be either directly calculated or simulated depending on model structure, is the basic reproductive number (otherwise know as the basic reproductive ratio), $R_0$, which represents the expected number of secondary infections that would be caused by one average infectious individual in an otherwise susceptible population \cite{Anderson1992}. By definition, if $R_0>1$ then you would expect an introduction of disease to lead to an epidemic, whereas if $R_0<1$ then you would expect disease to die out.

These compartmental models focus on the number, or proportion, of individuals who are susceptible, infected, and so on. However, with helminth infections it is much more common for biological measures of disease to be in terms of intensity rather than prevalence, and we would expect different dynamics to arise from two populations with the same prevalence level but different intensities of infection. As such, models of helminth transmission often consider the rate of change of mean worm burden (the average number of adult worms per individual in the population) \cite{Anderson1992}. Models therefore focus on the worm population dynamics rather than tracking human prevalence, usually with a term considering the rate of change of larval stages either in the environment or some intermediary host. A negative binomial distribution of worms is then usually assumed to calculate human prevalence directly from the modelled mean worm burden \cite{Anderson1992}.

The reproductive ratio, $R_0$, for helminths is described as the average number of female worms produced by one mature female worm during its reproductive life span \cite{Anderson1992}. Similarly to the SIR model, $R_0$ must be greater than one for disease to establish itself and it must be less than one for disease to die out. Interventions, such as vaccines, MDA or vector control can also be added to the model and the effective reproductive number, $R_e$, can be calculated. If $R_e$ is less than one then the interventions are sufficient to make the disease die out.

Helminth models also have a property that isn't seen in most other models of disease: an infected individual in which there are no fertilised female worms is not infectious and will not contribute to onward transmission \cite{WHOLF}. Due to the requirement of sexual reproduction in the parasite life cycle, models suggest there is a `break-point': a level of mean intensity below which transmission is unsustainable due to the effect this has on the probability of a male and female adult worm coexisting in the same host \cite{Hardwick2019}. Below this break-point population growth is severely restricted and the parasite population would be expected to decay to extinction \cite{Anderson1985}. 

\section{Aims}

The majority of the guidelines for control or elimination of NTDs involve broad recommendations for numbers of rounds and coverage of MDA, which demographic should be targeted and a description of the stopping criteria. There is some variation in these guidelines according to factors such as the co-endemicity of other diseases or failing to hit targets despite achieving recommended MDA coverage. For example, in areas endemic for loiasis it is not possible to use ivermectin, a key drug recommended in combination with other medicines to treat LF, due to the potential for severe adverse events \cite{WHO_lfguideline}. 

However, these very specific cases where different methods are recommended are insufficient to account for the wide variation we expect to see in intervention effectiveness \cite{Dean2016}. As the majority of NTDs have life-cycles that involve free-living stages or intermediary hosts, environmental conditions have the potential to either assist or undermine the efforts of interventions \cite{Gunawardena,Zhang2013,LiTeng2017}. There are also certain aspects of disease biology, such as the seasonal impact of weather conditions on onward transmission potential of developing \textit{Ascaris} eggs or the probability of one infectious mosquito infecting a human with LF, that we know surprisingly little about.

Models are an important tool that can help us investigate how important these unknowns are and work towards quantifying the impact they could have on intervention success. Through models we can explore the effect the varying biological parameters and simulate the outcomes of different types of interventions. However, it is also vital to ensure that the experimental evidence base used to parameterise our models is broad, accessible and reliable \cite{Davis2019}. Models can provide real-world recommendations, but are also vital to helping us identify key unknowns, ideally then prompting further experimental research and an improved understanding of the biological processes that drive transmission. We aim to investigate a few of the following knowledge gaps by developing novel models of these processes.

Firstly, it is widely accepted in the literature that STH infections have seasonal drivers that can cause transmission fluctuations \cite{Katakam2016,Mekonnen2019,Abubakari2018}. However, there are very few modelling studies that attempt to address this challenge. Studies that do exist are relatively recent and have considered seasonality through a maximum temperature for egg survival \cite{Truscott2016} or human migration \cite{Vegvari2019}. There have also been some recent studies on schistosomiasis, a helminth parasite with a prominent animal reservoir of infection, that have attempted to describe periodic forcing due to seasonal variation \cite{LiTeng2017,Huang2019,Gao2017}. Field studies have shown correlation between re-infection and seasonal factors, such as rainfall \cite{Gunawardena}, but modelling so far hasn't questioned the impact that seasonal fluctuations in these conditions may have on program success, particularly in combination with the timing of MDA. 

Considering the dynamics of parasitic stages outside the host is also a common area of weakness in LF models, which generally do not explicitly model the larval dynamics \cite{Chan1998,Plaisier1998}. Recent developments have included an equation for the larval dynamics \cite{irvine2015,Norman2000_epifil}, but the vector dynamics are not described in the model. A more detailed literature review reveals that there is also a general lack of experimental evidence for key biological parameters, such the the probability an infectious bite will lead to a new human infection \cite{Hairston1968,Jones2014} and the parasite aggregation \cite{Irvine2017_Mosquitobite}. In the context of elimination targets, uncertainty in these key parameters could be the difference between predicting success or failure. We aim to build a simple model of extinction to investigate how severe these effects might be.

Current models of LF also don't explicitly model vector dynamics, in particular the interaction with different vector control measures. Instead, presence of vector control is assumed to result in a proportion reduction in biting rates, reducing transmission \cite{irvine2015}. The general consensus across modelling groups is that vector control is unlikely to have a substantial additional benefit when used in combination with MDA \cite{Irvine2017_Tripledrug}. However, the simplifying assumptions made in these models may lead to incorrectly estimating the impact of vector control, particularly where there are field studies that demonstrate the power of measures such as long lasting insecticide-treated bednets (LLINs) against LF transmission \cite{rebollo2015,Odermatt2008,Njenga2011,Blackburn2006}. Following conclusions made about the importance of annual biting rate whilst investigating elimination dynamics, I believe these processes are potentially highly important to quantifying transmission dynamics. Using vector population modelling methods adapted from the literature, we aim to derive a model that captures the effects of vector control on mosquito population and infection measures.

Finally, we aim to demonstrate how an explicit vector control model could be used to inform xeno-monitoring methods. Xeno-monitoring is a non-invasive method of assessing transmission and infection levels in a population by sampling vectors and either dissecting or testing for disease DNA and has been considered a potential option for systematic program-based surveillance by the WHO for almost two decades \cite{WHO2002xeno}. However, there is still very poor understanding of how xeno-monitoring measures relate back to factors of public health importance, such as human prevalence \cite{Schmaedick2014}. Programmatic usage has generally involved simple detection of presence or absence, where presence of infection results in follow-up testing of the human population \cite{Chadee2002,Rao2016}, or to identify potential hotspots of transmission \cite{Swaminathan2017}, rather than as a proxy for human prevalence. Progress in quantification of this relationship could potentially lead to more robust and cost-effective tools for post-validation surveillance.

In the following chapters I will detail the work done during my PhD to address these aims. 

\section{Approach}

I will mainly focus on formulating novel models of transmission, building on models from the literature and exploring the biological basis for the processes described. In particular, I will focus on the impact that adding the explicit modelling of commonly excluded aspects of these processes has on model outcomes. Due to a lack of relevant epidemiological data my explorations will be largely theoretical and speculative, but I will discuss the relevance of the results and how these could be made more directly applicable in future studies.

In \textbf{Chapter 1}, I build a seasonal transmission model for \textit{Ascaris lumbricoides}, a type of STH. The model is based on a widely used deterministic differential equation helminth model that describes both the mean adult worm burden and free-living stages \cite{Anderson1992}. Using experimental data, I fit relationships between environmental factors, namely temperature and rainfall, and the development, viability and host uptake of infective larval stages. These relationships are then used to fit seasonally varying parameters for these processes to mean climate data for specific settings, such as South Korea and Nigeria.

Using approximate Bayesian computation (ABC) to fit the model to epidemiological data taken from the literature, I compare how seasonal effects vary across settings. Most importantly, I consider the impact these seasonal fluctations have on predicted mass drug administration outcomes and suggest how it may be possible to optimise MDA timing to maximise impact.

Still thinking about the importance of biological details that are often overlooked, I move on to investigate the biological unknowns surrounding LF transmission in \textbf{Chapter 2}. A number of these quantities, including vector to host transmission rates, have a shockingly poor and sparse evidence base in the literature. Using methods taken from branching process theory \cite{Watson1875}, I use these transmission cycle parameters to describe the expected number of secondary infectious cases caused by a single infectious individual in a population close to elimination targets. This can also be used to calculate the ultimate probability of extinction, where all chains of transmission die out.

I then use a univariate analysis to investigate the importance of these vector and parasite parameters to predictions and what effect the associated uncertainty could have on elimination programs. I demonstrate that factors such as the annual biting rate and the single hit probability of an infectious mosquito passing on a viable infection are highly influential to model outcomes. The effect of parameters that are expected to be of importance, such as the fecund life span of the adult worm in the host, is found to be mitigated by the relatively strong evidence base behind their quantification. I then go on to advise next steps for experimental studies in refining the estimates of some of these parameters.

However some parameters, such as the annual biting rate, are highly dependent on setting, which is largely the cause of the wide variations found in the literature. Variations in biting rate result in widely varying estimates of transmission intensity and potential, which could be an important factor when considering control methods.

In \textbf{Chapter 3}, I build an explicit model of vector dynamics and infection, including processes to reflect the utilisation of a range of vector control measures: LLINs, indoor residual spraying (IRS) and larvicides. I generalise a compartmental mosquito gonotrophic cycle model from the malaria literature \cite{Killeen2000} and parameterise the implementation of vector control using experimental field studies \cite{Ngufor2011,Kroeger1995}. I then expand this model to include SEI (susceptible, exposed, infectious) dynamics and, by considering an age-structured formulation, describe an equilibrium solution.

As mosquito life spans and dynamics are relatively fast compared to human infection length and life expectancy, I will assume that the fast dynamics in the mosquito population can be approximated by the quasi-equilibrium solution. I will then use this model, as well as analytical derivations of key epidemiological measures such as the effective reproductive number, to investigate the impact of different vector control combinations on the size, infection profile and transmission potential of the mosquito population. Although my main focus is LF, I will also discuss potential applications of the model to malaria.

I will also discuss how I integrated the vector model with a model of human LF infection and considered the impact of continuous LLIN usage across an extended period of time on human disease intensity, inspired by field studies that have shown bednets alone could potentially lead to elimination of transmission \cite{rebollo2015}.

In \textbf{Chapter 4}, I build on the model described in the previous chapter and derive a method for linking human prevalence and mosquito DNA prevalence for LF, as is attempted through molecular xenomonitoring (MX). I derive a probabilistic model for human prevalence based on vector sampling methods, utilising the inherent biases observed across different trap types to measure vector prevalence and the proportion of the population that are parous. Using this model I estimate vector sample sizes required for specified levels of precision.

I will also discuss the common method of pooling mosquitoes, which involves testing multiple mosquitoes at once, with a positive outcome indicating that at least one of the mosquitoes in that pool were carrying LF DNA. This is a potentially cost saving method, particularly when considering the large sample sizes required, but complicates the interpretation of survey results when attempting to look beyond presence or absence of disease. I derive the relationship between vector prevalence and pool prevalence (the proportion of pools that test positive) depending on the pool size and conduct first investigations into the associated loss of power when pooling mosquitoes as opposed to individual testing.

I conclude, in \textbf{Chapter 5}, by briefly discussing the negative binomial assumption of parasite aggregation and using a few toy examples to demonstrate why it is important to be careful when using these assumptions in practice.