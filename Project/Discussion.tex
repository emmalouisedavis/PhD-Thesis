
This thesis has focused on a range of questions across the landscape of neglected tropical diseases (NTDs), helminths and vector-borne transmission. The key overarching aim has been to investigate current models in the literature and expand the utility of these by including more detailed modelling of the underlying processes. A secondary, but equally important, goal was to highlight knowledge gaps and lay the basis for future research that will be beneficial towards disease control and elimination. I addressed these questions by developing novel models for \textit{A. lumbricoides} (a soil-transmitted helminth, STH) and lymphatic filariasis (LF) transmission using a variety of modelling methods. 

My results show that there are a number of scenarios where more detailed models would be appropriate to assist understanding of transmission and control. For example, I demonstrated in Chapter \ref{chap:STH} that the importance of seasonal mass drug administration (MDA) timing is highly dependent on local weather profiles, showing how temperate climates could be exploited to maximise program impact. In Chapters \ref{chap:ELIM} and \ref{chap:VEC} I showed that annual biting rate is a key determinant of elimination success and that vector control has the potential to substantially reduce transmission, in contradiction to previous modelling conclusions \cite{Irvine2017_Tripledrug}. In Chapter \ref{chap:XENO} I utilised an explicit model of mosquito feeding dynamics to calculate estimates of sample sizes required for the viable use of xenomonitoring as a tool for disease surveillance, which appear more feasible than suggested in the existing literature \cite{Schmaedick2014}.

The main implications of the seasonal model results in Chapter \ref{chap:STH} are that there may be a number of settings where MDA programs are a long way from achieving their potential maximum benefit. A relative improvement of 75\% across a four year period would represent a substantial gain in population health and disease control, particularly in settings that still see transmission after many rounds of MDA as benefits are expected to be multiplicative \cite{WHO2005_LF}. The recommendations prompted by these results are fairly simple and suggest that optimal treatment timing should coincide with environmental lows in the larval population \cite{Davis2018}. Similarly, the worst time for MDA was found to be during peaks in the infective larval population. As the relationship between temperature and development is reasonably well-characterised for ascaris eggs \cite{Wagner}, these periods should be relatively straightforward to determine.

The main bulk of the remaining chapters focused on aspects of the LF transmission cycle and their relationship with control and surveillance methods. Chapter \ref{chap:ELIM} has important connotations for future assessment and modelling of elimination dynamics. The identification of key knowledge gaps and recommendations for how they could be improved upon could potentially lead to substantially increased understanding of the disease \cite{Davis2019}, which could be instrumental in achieving elimination targets. Secondly, drawing awareness to the weaknesses in the experimental evidence base is vital to assuring models are realistic and well parameterised, highlighting the importance of critical assessment when taking parameters directly from the literature.

Findings from the mosquito model developed and analysed in Chapter \ref{chap:VEC}, show that adult-acting control measures are likely to have a much greater impact on transmission than larval-based interventions. Although larval control has been cited as a potentially viable method for reducing disease \cite{Kroeger1995}, these results demonstrate a multiplicative benefit of increasing coverage of adult-acting measures, compared to a linear benefit of increasing larvicide coverage. This implies that, unless vector population collapse is achieved through aggressive larvicide usage, programs would be better off focusing on long lasting insecticidal treated net (LLIN) and indoor residual spraying (IRS) interventions. The conclusion that high coverage (80\%) of LLIN usage alone could cause dramatic decreases in population mean worm burden across an order of 4-5 years. 

In addition, the recommendations for vector survey sample sizes and methods, presented in Chapter \ref{chap:XENO}, have far-reaching implications and potential impact. A statistically robust method of estimating the likely window of human prevalence from vector surveillance is critical to the feasibility of xenomonitoring as a public health tool and I have made substantial progress in this respect. The direct calculation of confidence intervals and guidelines on the sample sizes required for individually-sampled mosquitoes described here could lead to the development of a well-defined and applicable xeno-monitoring strategy that could be used by programs for surveillance post-validation of elimination as a public health problem.

My initial assessment that pooling vectors (up to and including pool sizes of 20) doesn't dramatically increase the error of measurements is a step towards making such strategies affordable and practically achievable for programs with fewer resources. I would expect that focusing on a low prevalence setting ($<$5\% microfilaria, mf, prevalence in humans) is the cause of this effect, making it ideal for elimination programs, but further analysis needs to be done to better understand this relationship.

I believe the consideration of aggregation assumptions in Chapter \ref{chap:AGG} is important, with results demonstrating that using a constant or linear form for the aggregation parameter, $k$, may be inappropriate for estimating prevalence from mean worm burden over time. However, this relatively short section of work serves only to demonstrate some potential pitfalls. Further work is necessary to characterise the extent of the limitations of the negative binomial assumption, as well as to develop a viable alternative if it is required.

My results are founded upon modelling assumptions. Parasite density dependent effects are neglected due to their reduced importance in low prevalence settings and other simplifications are assumed across the models, including constant pool sizes in the xenomonitoring analysis and no consideration of insecticide or chemotherapy resistance developing where control methods achieve high coverage.

Although it is important to take modelling assumptions into consideration when analysing results, I have demonstrated new methods for modelling helminth infections in humans and drawn some important conclusions. In some cases my results challenge the status quo, such as implying vector control usage may be more important than previously concluded in modelling studies. In other cases they confirm current understanding, for example by demonstrating why some regions may not see any seasonal variation in STH prevalence or intensity. In addition to these, I have also contributed some new understanding to the field, including a greater awareness of the importance of disease transmission parameters and novel suggestions for xenomonitoing error quantification.

The modelling undertaken during the work in this thesis is somewhat limited by the lack of general availability of detailed epidemiological data on the diseases discussed, both in terms of baseline and intervention measures. As such, the results are mainly theoretical and a key outcome is to highlight some areas where future modelling work, in collaboration with field observations, should be focused.

In particular, future modelling work is important around the dynamics of elimination of transmission of LF and how we can improve control methods for both LF and STH. Similar work investigating the impact of seasonal variation on transmission for other STH parasites would be important for deciding whether treatment programs should target their timing as the same drug is often used for multiple parasites, as well as extension of the present work to a wider range of settings and weather profiles. Greater clarity is also required on the characterisation of parasite aggregation within human populations under different external forces.

\chapter*{Concluding remarks}
\addcontentsline{toc}{chapter}{Concluding remarks}

The research reported in this thesis has investigated the impact of biological and environmental factors on neglected tropical disease (NTD) control, elimination and surveillance using mathematical and statistical modelling methods. I have demonstrated that challenging modelling assumptions and realising what we don't know can lead to deeper understanding of the processes involved, particularly where there is a paucity of data, and highlighted where further research is required. Through this work I have addressed some of the challenges faced by NTD control programs and my results contribute to the growing modelling evidence base within the field.